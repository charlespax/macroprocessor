%----------------------------------------------------------------------------------------
%	PACKAGES AND OTHER DOCUMENT CONFIGURATIONS
%----------------------------------------------------------------------------------------

\documentclass[
    letterpaper, % Page size: letterpaper (8.5x11 in, a4paper, executivepaper (7x10 in)
	fontsize=10pt, % Base font size
	twoside=true, % Use different layouts for even and odd pages 
                      % (in particular, if twoside=true, the margin column will be always on the outside)
	%open=any, % If twoside=true, uncomment this to force new chapters to start on any page, not only on right (odd) pages
	%chapterentrydots=true, % Uncomment to output dots from the chapter name to the page number in the table of contents
	numbers=noenddot, % Comment to output dots after chapter numbers; the most common values for this 
                          % option are: enddot, noenddot and auto (see the KOMAScript documentation for an in-depth explanation)
]{kaobook}

% Choose the language
\ifxetexorluatex
	\usepackage{polyglossia}
	\setmainlanguage{english}
\else
	\usepackage[english]{babel} % Load characters and hyphenation
\fi
\usepackage[english=british]{csquotes}	% English quotes

% Load packages for testing
\usepackage{blindtext} % Similar to lorem ipsum
%\usepackage{showframe} % Uncomment to show boxes around the text area, margin, header and footer
%\usepackage{showlabels} % Uncomment to output the content of \label commands to the document where they are used

% Load the bibliography package
\usepackage{kaobiblio}
\addbibresource{main.bib} % Bibliography file

% Load mathematical packages for theorems and related environments
\usepackage[framed=true]{kaotheorems}

% Load the package for hyperreferences
\usepackage{kaorefs}

\graphicspath{{examples/documentation/images/}{images/}} % Paths in which to look for images

\makeindex[columns=3, title=Alphabetical Index, intoc] % Make LaTeX produce the files required to compile the index

\makeglossaries % Make LaTeX produce the files required to compile the glossary
\newglossaryentry{immediateWord}{
	name={immediate word},
	description={is executed during compilation and interpretation}
}

\newglossaryentry{compileWord}{
        name={compile word},
        description={description}
}

% Glossary entries (used in text with e.g. \acrfull{fpsLabel} or \acrshort{fpsLabel})
\newacronym[longplural={Central Processor Units}]{cpuLabel}{CPU}{Central Processor Unit}
 % Include the glossary definitions

\makenomenclature % Make LaTeX produce the files required to compile the nomenclature


% Reset sidenote counter at chapters
%\counterwithin*{sidenote}{chapter}

% Load the package for byte field graphic generation
% https://ctan.org/pkg/bytefield
\usepackage{bytefield}

% Load the package for timing diagrams
% https://ctan.org/pkg/tikz-timing
\usepackage{tikz-timing}[2009/05/15]

% Load package for drawing diagrams
\usetikzlibrary{positioning, matrix, arrows.meta,calc}

% New Command for fixed width text
\newcommand{\fixed}[1]{\texttt{#1}}

%----------------------------------------------------------------------------------------
% Packages to consider
%----------------------------------------------------------------------------------------
% CircuiTikZ
% https://ctan.org/pkg/circuitikz
% The package provides a set of macros for naturally typesetting electrical and (somewhat 
% less naturally, perhaps) electronic networks. 


%----------------------------------------------------------------------------------------
% Notes
%----------------------------------------------------------------------------------------
% - This Reddit thread has many good links for understanding the inner workings of
%   the Forth system.
%   https://www.reddit.com/r/Forth/comments/8j6rsw/i_dont_understand_whats_happening_under_the_hood/
%
% - Check out preForth and seedForth. There's a video on youtube
%
% - Structure the book based on abstraction layers
%   * Instruction Set Architecture
%   * Machine Architecture
%   * Machine code
%   * Hardware Architecture
%   * Hardware virtualization
%   * ASIC implementation

%---------------------------------------------------------------------------------------

\begin{document}

%----------------------------------------------------------------------------------------
%	BOOK INFORMATION
%----------------------------------------------------------------------------------------

%\titlehead{The \texttt{kaobook} class}
%\subject{Use this document as a template}

\title[The Forth Enlightenment]{The Forth Enlightenment}
\subtitle{Hardware for the Forth programming language}

\author[Charles Edward Pax]{Charles Edward Pax}

\date{\today}

\publishers{Self-publishing}

%----------------------------------------------------------------------------------------

\frontmatter % Denotes the start of the pre-document content, uses roman numerals

%----------------------------------------------------------------------------------------
%	OPENING PAGE
%----------------------------------------------------------------------------------------

%\makeatletter
%\extratitle{
%	% In the title page, the title is vspaced by 9.5\baselineskip
%	\vspace*{9\baselineskip}
%	\vspace*{\parskip}
%	\begin{center}
%		% In the title page, \huge is set after the komafont for title
%		\usekomafont{title}\huge\@title
%	\end{center}
%}
%\makeatother

%----------------------------------------------------------------------------------------
%	COPYRIGHT PAGE
%----------------------------------------------------------------------------------------

%\makeatletter
%\uppertitleback{\@titlehead} % Header
%
%\lowertitleback{
%	\textbf{Disclaimer}\\
%	Warranty disclaimer here
%	
%	\medskip
%	
%	\textbf{Copyright}\\
%	Copyright \textcopyright 2023 Charles Edward Pax. All rights reserved, for now.
%	
%	\medskip
%	
%	\textbf{Colophon} \\
%	This document was typeset with the help of \href{https://sourceforge.net/projects/koma-script/}{\KOMAScript} 
%       and \href{https://www.latex-project.org/}{\LaTeX} using the \href{https://github.com/fmarotta/kaobook/}{kaobook} class.
%	
%	The source code of this book template is available at:\\\url{https://github.com/fmarotta/kaobook}
%	
%	\medskip
%	
%	\textbf{Publisher} \\
%	First printed in December 2023 by \@publishers
%}
%\makeatother

%----------------------------------------------------------------------------------------
%	DEDICATION
%----------------------------------------------------------------------------------------

%\dedication{
%	It would be nice to have a dedication to a person or group of people.\\
%	\flushright -- Charles Edward Pax
%}

%----------------------------------------------------------------------------------------
%	OUTPUT TITLE PAGE AND PREVIOUS
%----------------------------------------------------------------------------------------

% Note that \maketitle outputs the pages before here

\maketitle

%----------------------------------------------------------------------------------------
%	PREFACE
%----------------------------------------------------------------------------------------

%\chapter*{Preface}
\addcontentsline{toc}{chapter}{Preface} % Add the preface to the table of contents as a chapter

This is the preface page. This book is broken into two parts:

\begin{itemize}
	\item Hardware
	\item Software
\end{itemize}

The main goal of this project is to create a system that a user can visually verify down
to the transistor level. The physical layout of the circuit should correspond to the
schematic drawing layout where possible.

\begin{flushright}
	\textit{Charles Edward Pax}
\end{flushright}

%\index{preface}

%----------------------------------------------------------------------------------------
%	TABLE OF CONTENTS & LIST OF FIGURES/TABLES
%----------------------------------------------------------------------------------------

\begingroup % Local scope for the following commands

% Define the style for the TOC, LOF, and LOT
%\setstretch{1} % Uncomment to modify line spacing in the ToC
%\hypersetup{linkcolor=blue} % Uncomment to set the colour of links in the ToC
\setlength{\textheight}{230\hscale} % Manually adjust the height of the ToC pages

% Turn on compatibility mode for the etoc package
\etocstandarddisplaystyle % "toc display" as if etoc was not loaded
\etocstandardlines % "toc lines" as if etoc was not loaded

\tableofcontents % Output the table of contents

%\listoffigures % Output the list of figures

% Comment both of the following lines to have the LOF and the LOT on different pages
\let\cleardoublepage\bigskip
\let\clearpage\bigskip

%\listoftables % Output the list of tables

\endgroup


%%%==============================================================================%%%
%%%==============================================================================%%%
%%%                                                                              %%%
%%%    ███╗   ███╗ █████╗ ██╗███╗   ██╗    ██████╗  ██████╗ ██████╗ ██╗   ██╗    %%%
%%%    ████╗ ████║██╔══██╗██║████╗  ██║    ██╔══██╗██╔═══██╗██╔══██╗╚██╗ ██╔╝    %%%
%%%    ██╔████╔██║███████║██║██╔██╗ ██║    ██████╔╝██║   ██║██║  ██║ ╚████╔╝     %%%
%%%    ██║╚██╔╝██║██╔══██║██║██║╚██╗██║    ██╔══██╗██║   ██║██║  ██║  ╚██╔╝      %%%
%%%    ██║ ╚═╝ ██║██║  ██║██║██║ ╚████║    ██████╔╝╚██████╔╝██████╔╝   ██║       %%%
%%%    ╚═╝     ╚═╝╚═╝  ╚═╝╚═╝╚═╝  ╚═══╝    ╚═════╝  ╚═════╝ ╚═════╝    ╚═╝       %%%
%%%                                                                              %%%
%%%  https://patorjk.com/software/taag/#p=display&f=ANSI%20Shadow&t=MAIN%20BODY  %%%
%%%==============================================================================%%%
%%%==============================================================================%%%

\mainmatter % Denotes the start of the main document content, resets page numbering and uses arabic numbers
\setchapterstyle{kao} % Choose the default chapter heading style

\setchapterimage{seaside}
\setchapterpreamble[u]{\margintoc}
\chapter{Introduction}
\labch{intro}

%
% Key concepts and abilities
%==============================================================================
\begin{kaobox}[frametitle=In This Chapter]
you will learn
\begin{itemize}
	\item First topic
	\item Second topic
        \item Third topic
\end{itemize}

you will be able to
\begin{itemize}
        \item Task one
        \item Task two
        \item Task three
\end{itemize}
\end{kaobox}

% Introductory paragraph goes here
\blindtext

%
% Section: Purpose %
%==============================================================================
\section{Purpose}
\labsec{purpose}

The purpose of this book is to address the necessity of trustless 
computing or verifyable computing.

\begin{itemize}
	\item Don't trust, verify
	\item trustless computing
	\item trusting the compiler
\end{itemize}

Design princples: simplicity is a virtue.

%=================================%
% Subsection: Don't trust, verify %
%=================================%
\subsection*{Don't trust, verify}
In this section talk about blah blah blah.

%=================================%
% Subsection: Trustless computing %
%=================================%
\subsection*{Trustless computing}
Discuss trustless computing here.

%===================================%
% Subsection: Trusting the compiler %
%===================================%
\subsection*{Trusting the compilers}
In this subsection there will be a discussion of trusting the compiler.

%==================================%
% Subsection: Application: Bitcoin %
%==================================%
\subsection*{Application: Bitcoin}
These topics are relevant to Bitcoin because it relates to money. Don't
get hacket. Artificial Intelligence is coming for your money.


%
% Section: Hardware Overview %
%==============================================================================
\section{Hardware Overview}
\labsec{hardwareOverview}

No microcode. Implement logic in hardware. Complex optimization is bad. conceptual
simplicity. Simplicity is a cirtue.

\begin{description}
	\item[visual verification] view with your own eyes
	\item[simplicity] keep it simple to understand
        \item[repitation] repeat blocks
\end{description}

\subsection*{Stack computers}
\begin{itemize}
        \item Why stack computers?
        \item Parts of a stack computer
        \item Types of stack computers
\end{itemize}

Do we cover the instruction cycle here? Instruction timing?

%
% Section: Instruction Set Architecture %
%==============================================================================
\section{Instruction set Architecture}
\labsec{isa}
\index{instruction set architecture}

The instruction set architecture should be kept simple. Each bit encodes specific information.

%
% Section: Software Overview %
%==============================================================================
\labsec{softwareOverview}

The software should:

\begin{itemize}
	\item be simple
        \item reuse code blocks
        \item not require a compiler
        \item be hand-writable
\end{itemize}

Threaded code. No compiler. Forth.

\setchapterimage{seaside}
\setchapterpreamble[u]{\margintoc}
\chapter{Hardware}
\labch{hardware}

%
% Key concepts and abilities
%==============================================================================
\begin{kaobox}[frametitle=In This Chapter]
you will learn
\begin{itemize}
	\item First topic
	\item Second topic
        \item Third topic
\end{itemize}

you will be able to
\begin{itemize}
        \item Task one
        \item Task two
        \item Task three
\end{itemize}
\end{kaobox}

% Introductory paragraph goes here
\blindtext

%
% Section: Stack Computers
%==============================================================================
\section{Stack Computers}
We will discuss stack computers\sidecite{Koopman1989}.


\setchapterimage{seaside}
\setchapterpreamble[u]{\margintoc}
\chapter{Instruction Set Architecture (ISA)}
\labch{isa}

%
% Key concepts and abilities
%==============================================================================
\begin{kaobox}[frametitle=In This Chapter]
you will learn
\begin{itemize}
	\item First topic
	\item Second topic
        \item Third topic
\end{itemize}

you will be able to
\begin{itemize}
        \item Task one
        \item Task two
        \item Task three
\end{itemize}
\end{kaobox}

% Introductory paragraph goes here
\blindtext

%
% Section: ISA Explained
%==============================================================================
\section{ISA Explained}
Explain what an ISA is.

%
% Section: Instruction cycle
%==============================================================================
\section{Instruction cycle}
\blindtext

%
% Section: Instruction Format
%==============================================================================
\section{Instruction Format}

%
% Section: Instruction Timing
%==============================================================================
\section{Instruction Timing}
\blindtext

\begin{figure}[hbt!]
\begin{tikzpicture}[
    myline/.style={draw=green!40!black, thick},
    box/.style={myline, minimum height=1cm, minimum width=1cm, font=\sffamily, inner sep=.3333em}, >=Stealth]

    \matrix (CPU) [matrix of nodes, inner ysep=3mm, nodes=box, myline, column sep=1mm]
        {|(CU)|CU & |[draw=none]|CPU & |(ALU)| ALU \\};
    \node[box, above left=5mm of CPU] (input) {INPUT};

    \node[box, below left=5mm of CPU] (output) {OUTPUT};
    \node[box, at=(CPU|-output)] (memory) {MEMORY};
    \node[box, below left=5mm of memory, anchor=north] (storage) {STORAGE};

    \draw[<->, myline] (storage)-|(memory);
    \draw[->,myline] (input)-|(CU);
    \draw[->,myline] (CU)|-(output);

    \draw[<->, myline] ($(CU.south west)!.2!(CU.south east)$) coordinate (aux)--(aux|-storage.north);
    \draw[<->, myline] ($(CU.south west)!.8!(CU.south east)$) coordinate (aux)--(aux|-memory.north);
    \draw[<->, myline] ($(ALU.south west)!.2!(ALU.south east)$) coordinate (aux)--(aux|-memory.north);
\end{tikzpicture}\caption[Timing diagram]{This is an example timing diagram that will be replaced by the real thing.}
\end{figure}

\blindtext
\blindtext

\begin{figure}[hbt!]
\begin{tikztimingtable}
    RESET & l 29{H} \\
    Clock & 3{10{C}} \\
    Clock & 3{10{C}} \\
    Name & 3{hLLLLh} \\
    Signal & 3{z8D{Text}z} \\
    Signal & {z8D{Setup 00}z} {z8D{Setup 01}z} {z8D{Setup 10}z} {z8D{Setup 11}z} \\
    Signal & {z8D{PC> ,>MAR}z} {z8D{PM> ,>IR ,PC++}z} {z8D{PC> ,>MAR}z} {z8D{LC=IR[]}z} \\
\end{tikztimingtable}
\caption[Timing diagram]{This is an example timing diagram that will be replaced by the real thing.}
\end{figure}



\begin{figure*}[h!]
\begin{lstlisting}
        _______________________________________________________________________
RESET _/       

           _______         _______         _______         _______         ____
 CLK  ____/       \_______/       \_______/       \_______/       \_______/    
      ... | Execute 00    | Execute 01    | Execute 10    | Execute 11   |
          |  PC> ,>MAR    | PM> ,>IR ,PC++| PC> ,>MAR     | CL=IR[]      | ...
      ____         _______         _______         _______         _______     
~CLK      \_______/       \_______/       \_______/       \_______/       \____
  ... Setup 00    | Setup 01      | Setup 10      | Setup 11      | ...
      PC> ,>MAR   | PM> ,>IR ,PC++| PC> ,>MAR     | CL=IR[]       |
\end{lstlisting}
\caption[Timing chart]{Instructions are executed. The contyrol logic is configured in between each step. Timing chart.}
\end{figure*}


























\setchapterimage{seaside}
\setchapterpreamble[u]{\margintoc}
\chapter{The Forth Programming Language}
\labch{forth}

%
% Key concepts and abilities
%==============================================================================
\begin{kaobox}[frametitle=In This Chapter]
you will learn
\begin{itemize}
	\item First topic
	\item Second topic
        \item Third topic
\end{itemize}

you will be able to
\begin{itemize}
        \item Task one
        \item Task two
        \item Task three
\end{itemize}
\end{kaobox}

% Introductory paragraph goes here
\blindtext

%
% Section: The Forth Inner Interpreter
%====================================================================
\section{The Forth Inner Interpreter}
This section gives a general explaination\sidenote[]{Use sidenotes if necessary} 
of the chapter topic\sidecite{Loeliger1981}.

The word \lstinline|:| causes the innter interpreter to enter compile mode. The name of the
word is next, followed by the definition. The word \lstinline|;| causes the inner
interpreter to exit compilation and reenter interpretation mode.

Example:
\begin{lstlisting}[caption={Definition of \lstinline|dup| in Forth.}]
: square ( x -- x )
    dup
    *
    :
\end{lstlisting}

This, of course, can be more conveniently written on a single line
\begin{lstlisting}[style=kaolstplain,linewidth=1.5\textwidth]
: square ( x -- x )
    dup * ;
\end{lstlisting}

\section{Return Stack}
The top of the return stack holds the return address. When a function call is made, the return address
must be stored.  If the call command is executed at address x, the return address x+1 is pushed to the top of the return stack.

\section{Data Stack}
The data styack is where temporary data is stored.

If a system has only one hardware stack, it shoulld be used for the data stack. Data 
manipulations are more common than function calls and the hardware stack is faster 
than memory access. The call stack will need to be stored in memory.

%
% Section: The Forth Outter Interpreter
%====================================================================
\section{The Forth Outter Interpreter}
This an intermediate section. There will be several. each such section will cover a
discrete topic within this chapter.

\marginnote{Margin notes can be used to explain detail or add reminders that would
otherwise break the flow of the document. I think sidenotes are numbered while
margin notes are not.}

\blindtext

\setchapterimage{seaside}
\setchapterpreamble[u]{\margintoc}
\chapter{Template Chapter}
\labch{chaptertemplate}

%
% Key concepts and abilities
%==============================================================================
\begin{kaobox}[frametitle=In This Chapter]
you will learn
\begin{itemize}
	\item what is the first topic
	\item how to identify the second topic
        \item best practices for the third topic
\end{itemize}

you will be able to
\begin{itemize}
        \item perform the steps required to accomplish task one
        \item create a program that performs task two
\end{itemize}
\end{kaobox}

This introduction gives the reader any context necessary to understand the 
proceeding sections in this chapter.

The contents of the `In This Chapter' box should concrete. This should coorespond to
a `Chapter Review' section at the end of the chapter.

Illuminate any common obstacles for the reader. A few words of encouragement are also welcomed.

%
% Section: Heaps of Examples
%==============================================================================
\section{Heaps of Examples}
This section is fairly dense with examples. You can delete everything from here until
the end of the page.\sidenote[]{After you have saved this document as a new file.}

This section serves as an easy way to demonstrate\footnote{Like this footnote} multiple 
graphical elements\todo{Add a list of elements} that can be used in this book.

Keep chapters, sections, and sub sections capitalized.

%
% Subsection: Margin Stuff
%------------------------------------------------------------------------------
\subsection{Margin Stuff}
Generally, any text that would be written between parentheses would do well as
a margin note. Keep the flow going.

\marginnote{
	\begin{kaobox}[frametitle=Remember]
		a kaobox can be used in the margins for special reminders)
	\end{kaobox}
}

%
% Subsubsection: Sections in Depth
%
\subsubsection{Sections in Depth}
Sections within a chapter can be \Command{section}, \Command{subsection}, or
\Command{subsubsection}\marginnote{Notice that \Command{subsubsection} does not cause numbering.}.

%
% Subsection: Figures and Tables}
%------------------------------------------------------------------------------
\subsection{Figures and Tables}
\begin{lstlisting}[caption={
This Forth code does NOT generate \reftab{useless}
}]
: square  ( x -- x )
( Forth code that squares a number
    dup * ;
\end{lstlisting}

This is an example of a table contained withing the main column of text.

\begin{table}[ht]
\caption[A useless table]{A useless table.}
\labtab{useless}
\begin{tabular}{ c c c c }
    \toprule
    col1          & col2  & col3  & col4\\
    \midrule
    \multirow{3}{4em}{
    Multiple row} & 
        cell2 & cell3 & cell4\\ &
        cell5 & cell6 & cell7 \\ &
        cell8 & cell9 & cell10 \\
    \multirow{3}{4em}{
    Multiple row} & 
        cell2 & cell3 & cell4 \\ &
        cell5 & cell6 & cell7 \\ &
        cell8 & cell9 & cell10 \\
    \bottomrule
\end{tabular}
\end{table}


We can also have full-width tables.

\begin{table*}[h!]
    \caption{A wide table with invented data about three people living in the UK. Note that wide figures and tables are centered and their caption also extends into the margin.}
    \begin{tabular}{p{2.0cm} p{2.0cm} p{2.0cm} p{2.0cm} p{2.0cm} p{2.0cm} p{1.5cm}}
        \toprule
        Name    & Surname   & Job       & Salary           & Age   & Height    & Country \\
        \midrule
        Alice   & Red       & Writer    & 4.000 \pounds    & 34    & 167 cm     & England \\
        Bob     & White     & Bartender & 2.000 \pounds    & 24    & 180 cm     & Scotland \\
        Drake   & Green     & Scientist & 4.000 \pounds    & 26    & 175 cm     & Wales \\
        \bottomrule
    \end{tabular}
\end{table*}

And full-width images.

\blindtext

\begin{figure*}[h!]
	\includegraphics{seaside}
	\caption[A wide seaside]{A wide seaside, and a wide caption.
		Credits: By Bushra Feroz, CC BY-SA 4.0, \url{https://commons.wikimedia.org/w/index.php?curid=68724647}}
\end{figure*}

\begin{marginfigure}
	\includegraphics{output2}
	\caption[A Dot diagram]{A diagram produced by the Dot program and saved as a PNG file.}
	\labfig{marginmonalisa}
\end{marginfigure}


%
% Subsection: Citations
%------------------------------------------------------------------------------
\subsection{Citations}
Cite an author in the margin\sidecite{example1999}.

\blindtext


%
% Subsection: Glossaries and Indicies
%------------------------------------------------------------------------------
\subsection{Glossaries and Indicies}

%
% Subsection: Mathematics
%------------------------------------------------------------------------------
\subsection{Mathematics}
The box below was created using the \Environment{definition} environment. Boxes are also
provided by the \Environment{theorem}, \Environment{proposition}, \Environment{lemma},
\Environment{corollary}, \Environment{example}, \Environment{remark}, and \Environment{exercise} environments.

\begin{definition}
\labdef{exdeflabel}
Let $y$ be some function of $x$ such that $y = f(x)$
\end{definition}

where $x$ and $y$ are both variables in \refdef{exdeflabel}.



\blindtext

%
% Chapter Review
%==============================================================================
\section{Chapter Review}
\begin{kaobox}[frametitle=Chapter Review]
you have learned
\begin{itemize}
	\item what is the first topic
	\item how to identify the second topic
        \item best practices for the third topic
\end{itemize}

you are able to
\begin{itemize}
        \item perform the steps required to accomplish task one
        \item create a program that performs task two
\end{itemize}
\end{kaobox}







\setchapterimage{seaside}
\setchapterpreamble[u]{\margintoc}
\chapter{Code Listings}
\labch{codelistingsn}

\fixed{\textbf{!}}\hspace{2mm} ( $n$ \ $addr$ --- ) Store $n$ at $addr$.
\begin{lstlisting}
    : !  ( n addr -- )
        TOS> ,>MAR
        TOS> ,>PM
\end{lstlisting}

\fixed{\textbf{:}}\hspace{2mm} ( --- ) Colon definition. Create a new word.
Usage:  : <word> <...> ;
\begin{lstlisting}
    : :  ( -- )
\end{lstlisting}

\fixed{\textbf{;}}\hspace{2mm} ( --- ) Terminate a colon definition. Resume interpretation.
Usage: : <word> <...> ;
\begin{lstlisting}
    : ;  ( -- )
\end{lstlisting}

\fixed{\textbf{>r}}\hspace{2mm} ( n --- ) Move $n$ from data stack to return stack.
\begin{lstlisting}
    : >r  ( n -- )
        TOS> ,>RS
\end{lstlisting}

\fixed{\textbf{@}}\hspace{2mm} ( $addr$ --- $n$ ) Put on TOS the value $n$ stored at address $addr$.
\begin{lstlisting}
    : @  ( addr -- n )
        TOS> ,>MAR
        PM> ,>TOS
\end{lstlisting}

\fixed{\textbf{(create)}} 
\begin{lstlisting}
    : (create)  ( -- )
\end{lstlisting}

\fixed{\textbf{,}}\hspace{2mm} Allot one cell of dictionary space. Store n at that cell's address.
\begin{lstlisting}
    : ,  (n -- ) 
                 
\end{lstlisting}

\fixed{\textbf{blk}}\hspace{2mm} User variable. Located at address \fixed{addr} is the number
of the block being interpreted.
\begin{lstlisting}
    : blk  ( -- addr ) ( User variable. Located at address addr )
                       ( is the number )
                       ( of the block being interpreted )
\end{lstlisting}

\fixed{\textbf{here}}
\begin{lstlisting}
    : here  ( -- addr ) ( Put on TOS the address of the next free )
                        ( cell in the dictionary )
\end{lstlisting}

\fixed{\textbf{interpret}}\hspace{2mm} ( --- ) Interpret the input string.
\begin{lstlisting}
    : interpret  ( -- )
\end{lstlisting}

\fixed{\textbf{next}}\hspace{2mm} ( --- ) Compile the machine instructions which simulate the 
Forth machine's \fixed{next} function. Must be used at the exit point of each code or ;code
definition.
\begin{lstlisting}
    : next  ( -- )
\end{lstlisting}

\fixed{\textbf{r>}}\hspace{2mm} ( --- $n$ ) Move $n$ from return stack to data stack.
\begin{lstlisting}
    : r>  ( ( -- n )
        RS> ,>TOS
\end{lstlisting}



\appendix % From here onwards, chapters are numbered with letters, as is the appendix convention

\setchapterstyle{lines}
\labch{appendix}
%\blinddocument

\chapter{Template Appendix}

%
% Section: One section
%==============================================================================
\section{One section}

\blindtext


%
% Section: Another Section
%==============================================================================
\section{Another Section}

\blindtext


%----------------------------------------------------------------------------------------

\backmatter % Denotes the end of the main document content
\setchapterstyle{plain} % Output plain chapters from this point onwards



%----------------------------------------------------------------------------------------
%	BIBLIOGRAPHY
%----------------------------------------------------------------------------------------

% The bibliography needs to be compiled with biber using your LaTeX editor, or on the command line with 'biber main' from the template directory

\defbibnote{bibnote}{Works are listed in order of first citation.\par\bigskip} % Prepend this text to the bibliography
\printbibliography[heading=bibintoc, title=Bibliography, prenote=bibnote] % Add the bibliography heading to the ToC, set the title of the bibliography and output the bibliography note


Title:  Implementing the Forth Inner Interpreter in High Level Forth
Date:  2016-09
URL:  http://www.euroforth.org/ef16/papers/hoffmann.pdf
Abstract:  This document defines a Forth threaded code (inner) interpreter written entirely in
    high level standard Forth. For this it defines a specific threaded code structure of colon
    definitions, a compiler from high level Forth to this threaded code and a corresponding
    inner interpreter to execute it. This inner interpreter can run in a stepwise way and so gives
    the surrounding environment control of its execution behavior. A real time environment
    thus might slice the execution of threaded code in small pieces and provide an interactive
    command shell while still meeting its real time requirements.
References:
    [1] A synchronous FORTH framework for hard real-time control, U. Hoffmann and A. Read, euroForth 2016
    [2] Threaded Interpretive Languages: Their Design and Implementation, R. G. Loeliger, McGraw Hill, 1981
Notes:
   - This paper describes a Forth inner interpreter that runs inside a Forth system. This allows
   the Forth system to control the execution of the 'virtual' interpreter. This is relevant
   to Real-Time Operating Systems (RTOS).
   - This paper is not relevant to the MacroProcessor Forth system

Title:  Fig-Forth internals
Date: 2021-09
URL:  https://www.jimbrooks.org/archive/programming/forth/FIG-FORTH\_internals.php
Abstract:  My notes from studying and modifying 8086 assembly code of FIG-FORTH. Describes
    how FIG-FORTH interpreter executes internally, how interpreting and compiling are virtually 
    the same operation, ingenious solutions used to assemble FORTH interpreter. Also describes 
    modifying FIG-FORTH to have direct-threaded code (DTC). 
Notes:
    - These notes are likely to be useful in understanding more low-level information.
    - This is not a walkthrough of the Forth system
    -"When FORTH starts, it begins executing machine-code of physical CPU. FORTH registers 
    IP SP RP of FORTH virtual processor are initialized. System words COLD WARM ABORT are 
    executed which completes initialization. ABORT executes QUIT"
    - "During execution of COLD which starts a FIG-FORTH system, QUIT creates command 
    prompt. QUIT contains words QUERY INTERPRET. QUERY accepts commands, INTERPRET 
    interprets or compiles them. QUIT emits OK message after INTERPRET returns."

%----------------------------------------------------------------------------------------
%	NOMENCLATURE
%----------------------------------------------------------------------------------------

% The nomenclature needs to be compiled on the command line with 'makeindex main.nlo -s nomencl.ist -o main.nls' from the template directory

\nomenclature{$c$}{Speed of light in a vacuum inertial frame}
\nomenclature{$h$}{Planck constant}
\nomenclature{CFA}{See Code Field Address}
\nomenclature{LFA}{See Link Field Address}
\nomenclature{NFA}{See Name Field Address}
\nomenclature{PFA}{See Parameter Field Address}

\renewcommand{\nomname}{Notation} % Rename the default 'Nomenclature'
\renewcommand{\nompreamble}{The next list describes several symbols that will be later used within the body of the document.} % Prepend this text to the nomenclature

\printnomenclature % Output the nomenclature



%----------------------------------------------------------------------------------------
%	GLOSSARY
%----------------------------------------------------------------------------------------

% The glossary needs to be compiled on the command line with 'makeglossaries main' from the template directory

\setglossarystyle{altlist} % Set the style of the glossary (see https://en.wikibooks.org/wiki/LaTeX/Glossary for a reference)
\glsaddall % print all terms in the glossary, not only referenced words
\printglossary[title=Glossary, toctitle=Glossary,nonumberlist] % Output the glossary, 'title' is the chapter heading for the glossary, toctitle is the table of contents heading

Dictionary
===================

;  "Semi-colon"
Terminates a definition created by : .

1. Exits colon definition mode
2. Addes the definition to the dictionary
3. Enters interpretation mode

    : ; ( -- )
        ( definition goes here ) ;


NEXT  "Next"
Terminates a code-word and jumps to the next instruction.

1. Fetch the Code Field Address (CFA),
   which is stored in a cell,
   which is pointed to by the Forth Instruction Pointer (IP)
2. Increment the Forth IP,
   which will then point to the proceeding cell,
   which contains the next Forth instruction
3. Jump to the fetched CFA

    : NEXT ( -- )
        RS> ,>PC ,R--


Notes:
   - What is a cell? I think this is address of the code field of the
   proceeding Forth word. The IP always points to the next CFA to be
   executed.
   - Does the cell located at the CFA hold a machine instruction or
   is there a Forth dictionary word there? I think the contents of the
   cell pointed to by the CFA in the IP is always a machine instruction.
   I think the word INTERPRET handles the task of looking through the
   dictionary and determining the CFA of that word. The CFA should be
   the first cell of a word's code field, which can be one or more cells.

















%----------------------------------------------------------------------------------------
%	INDEX
%----------------------------------------------------------------------------------------

% The index needs to be compiled on the command line with 'makeindex main' from the template directory

\printindex % Output the index



%----------------------------------------------------------------------------------------
%	BACK COVER
%----------------------------------------------------------------------------------------

% If you have a PDF/image file that you want to use as a back cover, uncomment the following lines

%\clearpage
%\thispagestyle{empty}
%\null%
%\clearpage
%\includepdf{cover-back.pdf}

%----------------------------------------------------------------------------------------

\end{document}
