\setchapterimage{seaside}
\setchapterpreamble[u]{\margintoc}
\chapter{The Forth System}
\labch{chaptertemplate}

%
% Key concepts and abilities
%==============================================================================
\begin{kaobox}[frametitle=In This Chapter]
you will learn
\begin{itemize}
	\item what is the first topic
	\item how to identify the second topic
        \item best practices for the third topic
\end{itemize}

you will be able to
\begin{itemize}
        \item perform the steps required to accomplish task one
        \item create a program that performs task two
\end{itemize}
\end{kaobox}

This introduction gives the reader any context necessary to understand the 
proceeding sections in this chapter.

The contents of the `In This Chapter' box should concrete. This should coorespond to
a `Chapter Review' section at the end of the chapter.

Illuminate any common obstacles for the reader. A few words of encouragement are also welcomed.

%
% Section: Things Forth Does
%==============================================================================
\section{Things Forth Does}
\subsection{System Operations}
Note: the text interpreter handles input from terminal and files. It also 
handles compilation

\subsubsection{Initialize system}
Copy values into the correct location in memory. Set register values. Example: 
Read from memory the designated address for the stack pointer and load that 
into a particular register.

\subsubsection{Interpret input stream}
The outer interpreter (AKA text interpreter). The input stream could be a 
keyboard buffer, serial buffer, or other source.

\subsubsection{Execution}
'Innter interpreter.' 'Address interpreter.' Track which word is being executed
and which to execute next. Jumping to machine code. Controls the flow of 
execution.

\subsubsection{Error detection and handling}
Usually by clearing the stack and returning control to the outer interpreter.

%
% Section: Forth System Parameters
%==============================================================================
\section{Forth System Parameters}
Data Stack Pointer (DS), Return Stack Pointer (RS), Interpreter Pointer(IP),
Word Pointer (W).

Data Stack Pointer. Holds the address of the top cell of the data stack. SP can be in memory or in a register. Register is better. This paragraph is true for systems that have the stack in RAM.

The data stack can be held in memory or in hardware stack provided by the CPU. The same is true of the return stack.

When using a hardware stack provided by the CPU, the Top of Stack (ToS) element is typically the only element that can be accessed by the data bus. Some microprocessors provide access to multiple cells. Three cells are a good number (Stack Computers?).

Return Stack Pointer (RS). Similar to the data stack. The data stack pointer and the data stack itself can beach be stored in memory. RS can be a hardware stack, but doing so does not provide the same performence gain as doind so with the data stack (source?).

During execution, a word may use the Return Stack for temkporary storage. Be careful!

Interpretative Pointer (IP). 'instruction pointer?'. Points to the code field (?) of the next word to be executed. Does thi depend on the system implementation? Can be stored in memory or a register.

Word Pointer (WP). Points to the word currently being executed. (Name field? Code field?).
% Chapter Review
%==============================================================================
\section{Chapter Review}
\begin{kaobox}[frametitle=Chapter Review]
you have learned
\begin{itemize}
	\item what is the first topic
	\item how to identify the second topic
        \item best practices for the third topic
\end{itemize}

you are able to
\begin{itemize}
        \item perform the steps required to accomplish task one
        \item create a program that performs task two
\end{itemize}
\end{kaobox}






