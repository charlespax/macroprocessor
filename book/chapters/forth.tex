\setchapterimage{seaside}
\setchapterpreamble[u]{\margintoc}
\chapter{The Forth Programming Language}
\labch{forth}

%
% Key concepts and abilities
%==============================================================================
\begin{kaobox}[frametitle=In This Chapter]
you will learn
\begin{itemize}
	\item First topic
	\item Second topic
        \item Third topic
\end{itemize}

you will be able to
\begin{itemize}
        \item Task one
        \item Task two
        \item Task three
\end{itemize}
\end{kaobox}

% Introductory paragraph goes here
\blindtext

%
% Section: Dictionay Entry Format
%====================================================================
\section{Dictionary Entry Format}

Each word in Forth is defined by a \gls{dictionaryEntry}
\gls{immediateWord}.



The dictionary entry format is dependent upon the Forth implmentation, but a typical
16-bit Forth system will implement the structure described in \reffig{dictionaryEntryFormat}.


\todo{Add headings for these three columns: Address, Contents, Description}
\begin{figure}[hbt!]
\begin{bytefield}[bitwidth=auto]{16}
    \begin{rightwordgroup}[rightcurly=.]{\textbf{Notes}}
            \begin{leftwordgroup}[leftcurly=.]{\textbf{Address}}
                \wordbox[]{1}{\textbf{Content}}
    \end{leftwordgroup}
\end{rightwordgroup} \\
    \bitheader[endianness=big]{15,14,0} \\
\begin{rightwordgroup}{name field}
    \begin{leftwordgroup}[leftcurly=.]{Name Field Address (NFA)}
        \bitbox{1}{$p$} &
        \bitbox{15}{$n$}
    \end{leftwordgroup} \\
    \begin{leftwordgroup}[leftcurly=.]{NFA $+ 1$}
        \bitbox{16}{$name_0$}
    \end{leftwordgroup} \\
    \wordbox[]{1}{$\vdots$} \\[1ex]
    \begin{leftwordgroup}[leftcurly=.]{NFA $+ n + 1$}
        \bitbox{16}{$name_n$}
    \end{leftwordgroup}
\end{rightwordgroup} \\
\begin{rightwordgroup}[rightcurly=.]{link field}
    \begin{leftwordgroup}[leftcurly=.]{Link Field Address (LFA)}
        \wordbox{1}{link pointer}
    \end{leftwordgroup}
\end{rightwordgroup} \\
\begin{rightwordgroup}[rightcurly=.]{code field}
    \begin{leftwordgroup}[leftcurly=.]{Code Field Addresss (CFA)}
        \wordbox{1}{code pointer}
    \end{leftwordgroup}
\end{rightwordgroup} \\
\begin{rightwordgroup}{parameter field}
    \begin{leftwordgroup}[leftcurly=.]{Parameter Field Address (PFA)}
        \wordbox{1}{$parameter_0$}
    \end{leftwordgroup} \\
    \wordbox[]{1}{$\vdots$} \\[1ex]
    \begin{leftwordgroup}[leftcurly=.]{PFA $+ c$}
    \wordbox{1}{$parameter_c$}
    \end{leftwordgroup}
\end{rightwordgroup}
\end{bytefield}

\caption[Dictionary entry format]{Dictionary entry format.}
	\labfig{dictionaryEntryFormat}
\end{figure}

where \\
\todo{how are these items sorted? Alphabetical? Order of appearance?}
NFA is the address pointing to the first cell in the word's name field, \\
LFA is the address pointing to the word's link pointer field, \\
CFA is the address pointing to the word's code pointer field, \\
PFA is the address pointing to the first cell in the word's parameter field, \\
$p$ is the precedence bit (0,1), \\
$n$ is the number of characters in the word's name (e.g. DUP$_n = 3$), \\
$name_n$ is the $n^{th}$ character in the name, \\
$c$ is the number of cells in the parameter field, and \\
$parameter_c$ is the $c^{th}$ cell in the parameter field.

\marginnote{This text uses the term \gls{parameterField}. Other resources may use the term \gls{dataField} or other term.}

Each dictionary entry has a link pointer field, which holds an memory address pointing to the 
preceeding dictionary entry. By design, when the system searches for a word it starts 
at the end of the dictionary and works its way to the beginning by following these 
pointers. We call these addresses link pointers because they link the words together and to
distinguish them from pointers with other purposes.

Each \gls{dictionaryEntry} in the \gls{dictionary} will share the same format for the \gls{nameField}, \gls{linkField}, and \gls{codeField}. The \gls{parameterField} format depends on the word type. Constants, variables, colon-defined words, and any other type of defining word can have its own uniquie parameter field format.

The link pointer of a dictionary entry points to the name field of the previous word. The value of a link pointer is equal to the NFA of the pervious dictionary entry.
The link pointer is an address pointing to the first cell of the previous dictionary entry's name field. The LFA of a dictionary entry contains the NFA of the previous dictionary entry.i

Code pointer contains an address. This address is the "address of the instruction first executed when this type of word is executed." Starting Forth p. 222

Variable words. The code pushes the variable address onto the stack.

Constant words. The code pushes the contents of the constant onto the stack.

Colon defined words. The code executes the rest of the words in the colon definition.

In all the cases the code pointed to is called "tun-time code."

All words of a given type have the same code pointer.

code address is the address of the machine instruction the interpreter will load for
execution, and \\
data is any data that is required for execution.
todo[noline]{Decide if the name field characters are 16-bit or 8-bit in the final implementation.}
\marginnote{The code address field is also known as the code pointer field. The data field
is also known as the parameter field.}

Precedence bit. 0 - address of the word gets compiled normally. 1 - The word is executed 
during compilation.

Are the characters in the name field limited to ASCII? Is there any technical reason they
could not be any arbitrary 16-bit value? Maybe null would be a problem.

Leo Brodie's Starting Forth\sidecite{Brodie1981} is good reading for this.

%
% Section: The Forth Inner Interpreter
%====================================================================
\section{The Forth Inner Interpreter}
This section gives a general explaination\sidenote[]{Use sidenotes if necessary} 
of the chapter topic\sidecite{Loeliger1981}.

The word \lstinline|:| causes the innter interpreter to enter compile mode. The name of the
word is next, followed by the definition. The word \lstinline|;| causes the inner
interpreter to exit compilation and reenter interpretation mode.

Example:
\begin{lstlisting}[caption={Definition of \lstinline|dup| in Forth.}]
: square ( x -- x )
    dup
    *
    :
\end{lstlisting}

This, of course, can be more conveniently written on a single line
\begin{lstlisting}[style=kaolstplain,linewidth=1.5\textwidth]
: square ( x -- x )
    dup * ;
\end{lstlisting}

\section{Return Stack}
The top of the return stack holds the return address. When a function call is made, the return address
must be stored.  If the call command is executed at address x, the return address x+1 is pushed to the top of the return stack.

\section{Data Stack}
The data styack is where temporary data is stored.

If a system has only one hardware stack, it shoulld be used for the data stack. Data 
manipulations are more common than function calls and the hardware stack is faster 
than memory access. The call stack will need to be stored in memory.

For "colon words" (words compiled with the : word), the data field is a series of cells, each cell
containing an address pointing to a word in the Forth dictionary.

The address interpreter (inner interpreter) copies the address stored in the address field
into the interpreter pointer then executes the code at that address.

The interpreter does not keep track of or know about the type of word it is working with. The code linked to by the CFA is responsible for behaving appropriately.

%
% Section: The Forth Outter Interpreter
%====================================================================
\section{The Forth Outter Interpreter}
This an intermediate section. There will be several. each such section will cover a
discrete topic within this chapter.

\marginnote{Margin notes can be used to explain detail or add reminders that would
otherwise break the flow of the document. I think sidenotes are numbered while
margin notes are not.}

\blindtext
