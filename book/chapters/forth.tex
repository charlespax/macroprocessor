\setchapterimage{seaside}
\setchapterpreamble[u]{\margintoc}
\chapter{The Forth Programming Language}
\labch{forth}

%
% Key concepts and abilities
%==============================================================================
\begin{kaobox}[frametitle=In This Chapter]
you will learn
\begin{itemize}
	\item First topic
	\item Second topic
        \item Third topic
\end{itemize}

you will be able to
\begin{itemize}
        \item Task one
        \item Task two
        \item Task three
\end{itemize}
\end{kaobox}

% Introductory paragraph goes here
\blindtext

%
% Section: The Forth Inner Interpreter
%====================================================================
\section{The Forth Inner Interpreter}
This section gives a general explaination\sidenote[]{Use sidenotes if necessary} 
of the chapter topic\sidecite{Loeliger1981}.

The word \lstinline|:| causes the innter interpreter to enter compile mode. The name of the
word is next, followed by the definition. The word \lstinline|;| causes the inner
interpreter to exit compilation and reenter interpretation mode.

Example:
\begin{lstlisting}[caption={Definition of \lstinline|dup| in Forth.}]
: square ( x -- x )
    dup
    *
    :
\end{lstlisting}

This, of course, can be more conveniently written on a single line
\begin{lstlisting}[style=kaolstplain,linewidth=1.5\textwidth]
: square ( x -- x )
    dup * ;
\end{lstlisting}

\section{Return Stack}
The top of the return stack holds the return address. When a function call is made, the return address
must be stored.  If the call command is executed at address x, the return address x+1 is pushed to the top of the return stack.

\section{Data Stack}
The data styack is where temporary data is stored.

If a system has only one hardware stack, it shoulld be used for the data stack. Data 
manipulations are more common than function calls and the hardware stack is faster 
than memory access. The call stack will need to be stored in memory.

%
% Section: The Forth Outter Interpreter
%====================================================================
\section{The Forth Outter Interpreter}
This an intermediate section. There will be several. each such section will cover a
discrete topic within this chapter.

\marginnote{Margin notes can be used to explain detail or add reminders that would
otherwise break the flow of the document. I think sidenotes are numbered while
margin notes are not.}

\blindtext
